% Introduction: use a mixture of present and past tense; 
% the present tense is applied when you are talking about something that is always true; 
% the past tense is used for earlier research efforts, either by your own or by another group. 
% If the time of demonstration is unknown or not important, use the present perfect. 
% For the concluding statements of your introduction use the simple past; 
% you may use the past perfect, when you talk about something that was true in the past but is no longer so.
\section{Introduction} \label{section:introduction}
This thesis is framed within the general context of Artificial Intelligence (AI), 
focusing on the development of swarm intelligence (SI) in multi-agent games.
Early AI research concerned itself with the development of computational systems capable of displaying human-level intelligence.
The aim was to apply this intelligence in tasks of problem solving and decision making.
Such tasks were presented to the machines as a set of formal mathematical notations, 
which were able to be solved by means of symbol manipulation.
Due to this highly formalized framing of the problems, 
early AI was able to succeed in a large number of tasks.

Over the last couple of decades, the field has been thriving, 
with development happening at an increasing pace.
The efforts of AI research have not been limited to academia, 
as an increasing number of areas, unrelated to computer science, have adopted AI techniques, 
in the hope of increased performances.
The rapid advancements can be attributed to two main driving forces.
First and foremost, it is due to novel algoirthsm devides by the research community.
Secondly, the computational power of hardware, both in general and available to research, has increased, 
which allowed researchers to try increasingly more complex solutions.
These factors have allowed researchers to develop machines that have reached human-level intelligence and beyond in many fields, 
such as game-playing.

\textbf{Games}, in particular board games, have long been considered as the perfect research environment for AI methods.
This is partly due to them being a complex human activity.
However, the main reason is that they offer environments which are formal and highly constrained, 
suitable for decision making tasks. 

Game AI (GAI) is a sub-field of AI, which came to be quickly after AI itself.
It is a broad field, which is concerned with the challenge of developing AI for playing games at a high level, 
but also ventures into high level application, such as the automated generation of game aspects.

With the simultaneous rapid development of AI and the introduction of video games, 
the area of research has seen an overwhelming number of advancements in the last couple of years.
An early breakthrough occurred in the 1950s, when 
Shannon \cite{shannon1950xxii} 
developed a program for playing chess.
Other multi-player board games have been researched, for example, 
Othello \cite{buro2002improving}, 
Hex \cite{anshelevich2002hierarchical}, 
Shogi \cite{iida2002computer}, 
Go \cite{muller2002computer}, 
Backgammon \cite{tesauro2002programming}, 
Scrabble \cite{sheppard2002world} and 
Checkers \cite{chellapilla1999evolving}. 
Perhaps the most well-known feat in this field, 
was the victory of Deep Blue over world chess champion Garry Kasparov in 1997.
Another major achievement occurred in 2016 when DeepMind's AlphaGo defeated Lee Sedol in the game Go.

Numerous AI methods have been employed to tackle challenges in games, 
which include search algorithms, reinforcement- and deep learning, among others.
Many of these application, however, are concerned with the performance of a singular agent.

An emerging field, which explores the design of multi-agent systems, is swarm intelligence.
It is heavily inspired by the observation of collective behaviour of social insects, 
such as ants and bees.

Ants, for instance, are shown to use pheromone trails to communicate indirectly with each other, 
in tasks such as foraging food. 
Bees, on the other hand inform others in the swarm of new locations with an abundance of food, 
in a direct fashion through a specific dance.
The collective behaviour is not limited to foraging alone, 
as social insects are also known to build nests in cooperation.

Accordingly, other animal societies also display a form of collective behaviours, 
as can be seen in flocks of birds and schools of fish.

Although the individual members of the swarm are simple beings, 
together they are able to carry out complex tasks.
A key aspect of the collective behaviours, 
is that it emerges from relatively simple actions and interactions between individuals.

As such, these decentralized collective behaviours have caught the eye of AI researchers.
Thereforem, the design of applications in SI adhere to similar constraints.
As opposed to more traditional approaches, 
swarm intelligence does not make use of a sophisticated global controller to govern the behaviour of the system.
Instead, just as in nature, individual controllers, 
carrying out unsophisticated behaviours are employed, 
in order to achieve the desired behaviour by meand of cooperation.

\subsection{Research goals} \label{subsection:goals}
In recent years, the application of swarm intelligence in games has seen more research focus.
However, most research focuses on a limited number of traditional tasks that are to be performed by a swarm.
Having a more complex environment, 
with many actions available to the individual agents and the swarm has not seen much research as of yet.

This thesis will address the following research goals

\begin{enumerate}
    \item We aim to motivate work on swarm intelligence (SI) in video games
    \item We aim to apply SI to a novel developed multi-agent survival game
    \item We aim to focus on self-organized behaviour for task allocation
\end{enumerate}

\subsection{Thesis structure} \label{subsection:structure}

The thesis is organised as follows. 
In chapter \ref{section:background} background information is presented.
The chapter begins with a historic overview on Game AI, highlighting methods and major achievements.
Furthermore, we show the difference between academia and industry with regards to Game AI.
Additionally, the genre of Real Time Strategy (RTS) games is expanded upon.
The importance of projection in games is next touched upon.

Next, we define the basic concepts of swarm intelligence.
These include basic behaviours such as \textbf{foraging}, \textbf{flocking}, and \textbf{aggregation}.
This is followed by a description of self-organized behaviour, in particular task allocation.
The driving force behind actions in a swarm, Finite State Machines (FSM) is explained next.

Next in chapter \ref{section:methodology} the methods are presened. 
First, the game objectives are described. 
Next, the methods to build the game are described, which include the environment, camera and physics.
\textcolor{blue}{---
Further elements of the game are described in detail such as agents, enemies, and items.
---}
Finally, we describe the evaluation metric.

The results are presened in chapter \ref{section:results}, which are discussed in chapter \ref{section:discussion}.
The discussion will delve into the obtained results, coupling them to the research question which were defined in Section \ref{subsection:goals}.
Furthermore, we will look into the limitations of the proposed methods and will offer suggestions for future work.

Finally, the thesis is concluded in chapter \ref{section:conclusion}.
