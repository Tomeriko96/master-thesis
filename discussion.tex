\section{Discussion} \label{section:discussion}
% Discussion: use the simple past for your own findings 
% and the perfect tense for cited information; 
% the present tense is also acceptable, 
% if you prefer that one (in such statements as ‘We can conclude that …’.
In this work, we have presented a novel engine for the development of swarm intelligence.
Self-organizating behaviours in the context of task allocation has been demonstrated.

\textcolor{purple}{
    \blindtext
    \blindtext
}
\textcolor{purple}{
    \blindtext
    \blindtext
}

Refer to chapter \ref{subsection:goals} that defines the goals of our thesis. 
We have achieved all of our goals in this thesis. 
Therefore, we can identify the following contributions to research

\begin{enumerate}
    \item In chapter \ref{section:background} we motivated work on swarm intelligence in video games by highlighting relevant work in the field of Game AI.
    In particular we looked into real time strategy games.
    \item In chapter \ref{section:methodology} we documented the methods to develop a multi-agent survival game engine, in which swarm intelligence can be implemented.
    \item in chapter \ref{subsection:agent} we showed how Finite State Machines could be used to implement self-organized behaviour in a swarm of individually controlled agents, with a focus on task allocation.
    % \item In chapter \ref{subsection:evaluation_metric} we also showed how evolutionary computing can be applied to swarm intelligence.
\end{enumerate}

\subsection*{Motivating work on swarm intelligence}
\textcolor{purple}{
    \blindtext
    \blindtext
}

\subsection*{The game engine}
\textcolor{purple}{
    \blindtext
    \blindtext
}

\subsection*{Self-organized behaviour}
\textcolor{purple}{
    \blindtext
    \blindtext
}

% \subsection*{Evolving the swarm}
\textcolor{purple}{
    \blindtext
    \blindtext
}

\subsection{Limitations} \label{subsection:limitations}
% The hardware on which the simulation was developed was not optimal.

\subsection*{Motivating work on swarm intelligence}
\textcolor{purple}{
    \blindtext
    \blindtext
}

\subsection*{The game engine}
The main limitation presented by the game engine, is the difficulty in scaling.
Pygame is not intended to be used to simulate a large number of agents, and as such it showed, 
as with increasing numbers, the game became much slower and less reactive.
Ultimately, this is acceptable, as the engine developed is intended for academic purposes solely, 
however for more elaborate research, either hardware with stronger capabilities is needed, 
or the game needs to be ported to engines more capable of handling multi-agent environments.

% One could look into engines such as unity, 
\textcolor{purple}{
    \blindtext
    \blindtext
}

\subsection*{Self-organized behaviour}
\textcolor{purple}{
    \blindtext
    \blindtext
}



\subsection{Future work} \label{subsection:future_work}
Future work concerning the proposed framework, should focus on adding more complexity to the simulation.
This could be achieved by adding more challenging elements to the environment, such as bodies of water, bridges and fire.
Furthermore, the use of procedural content creation could prove useful, especially in training as the current implementation has a predictable environment due to the perlin noise.

\textcolor{purple}{
    \blindtext
    \blindtext
}