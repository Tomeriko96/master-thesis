% Abstract: generally, use the simple past 
% (or for a concise introductory phrase the present perfect); 
% for general statements and facts use the present tense.
\begin{abstract}

    
    Artificial Intelligence (AI) can be thought of as the study of machines that are capable of solving problems that require human level intelligence.
    It is a field in computer science which has seen a resurgence over the last few decades, both in academia and industry.
    Game AI (GAI) has long held the interest of researchers, with it being referred to as the tested for developing AI.
    Since the early 1950s, games have been a central theme of AI.
    Games address challenges that have great commercial, social, economic and scientific interest.
    Classical board games, such as chess, were long considered the drosophila of GAI, 
    due to their formal and highly constrained, yet complex, decision making nature.
    Over the years, researchers have achieving great results, achieving human level performances and beyond.
    As such, for decades research has been conducted on game playing.
    With the introduciton of video games, the efforts in GAI has increased rapidly.
    The complexity introduced by these games have been used to develop general AI.
    In video games with multiple agents, such as in Real Time Strategy (RTS) games, in combination with infinite game states, have added to the complexity.
    Novel AI solutions have been researched to address these challenging games.
    Swarm Intelligence (SI), a subcategory of AI, focuses on problem solving by means of collective behaviour of decentralized individual agents.
    Algorithms in SI, are inspired by the collective behavior of social organisms.
    The main principle of the collective behavior, is that it emerges from relatively simple actions between the indiviuals.
    \textcolor{blue}{---
    In this work, outlined is the general framework of interacting agents behaving as a swarm group in order to survive in a dangerous environment.
    The environment contains resources and threats, which are used to demonstrate classic swarm behaviours such as foraging.
    ---}
    Action selection of the agents are performed by means of Finite State Machines (FSM).
    The focus is put on the ability, of said swarm, to self-organize and distribute tasks among themselves.
    The obtained results show that the implementation has resulted in the expected behaviour to be emulated.
    This self-organized behaviour has lead to high level of fitness in the swarm population.
    
\end{abstract}